% Introduction
\section{Introduction}
This report explores the application of Lasso regression, a linear regression technique with regularization, and kNN regression, a non-parametric method, to forecast daily DC power output from a solar plant. The study utilizes a dataset containing historical solar power generation data and corresponding weather information, including temperature and irradiation. By leveraging these features and incorporating lagged variables, the models aim to capture temporal dependencies and patterns in the data, leading to improved prediction accuracy.

The project evaluates the performance of both univariate and multivariate Lasso regression models, as well as kNN models with varying numbers of neighbors. The univariate models rely solely on historical power data, while the multivariate models incorporate additional weather features. The impact of using hourly data versus daily averages on prediction accuracy is also investigated.

This report is structured as follows: Section 2 provides a background and literature review on solar power prediction and discusses various approaches explored in previous studies, including statistical methods, machine learning techniques, and deep learning models. Works employing Lasso and KNN for one-day-ahead predictions are also explored. Section 3 details the methodology employed, including data preprocessing, feature engineering, model development, and evaluation metrics. Section 4 presents the results and discussion, comparing the performance of different models and analyzing feature importance. Finally, Section 5 concludes the report with a summary of findings and suggestions for future work.