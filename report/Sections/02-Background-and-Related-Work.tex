\section{Background and Related Work}
Accurate prediction of solar power generation is essential for:
\begin{itemize}
    \item \textbf{Efficient Grid Management:} Balancing energy supply and demand.
    \item \textbf{Renewable Energy Integration:} Maximizing the use of solar power.
    \item \textbf{Economic Benefits:} Optimizing energy trading and reducing reliance on fossil fuels.
\end{itemize}

Various approaches, including statistical methods and machine learning techniques, have been explored for solar power prediction.

\citet{zafarani2018assessing} delved into the significance of weather data in solar power prediction by analyzing its impact on photovoltaic power generation forecasting accuracy. They identified key weather parameters influencing solar power output, emphasizing the importance of incorporating weather information for accurate predictions. The study employed various machine learning models, including support vector machines and artificial neural networks, to predict solar power generation and assessed the relative importance of different weather features.

\citet{Alanazi_2018} introduced a day-ahead solar forecasting model leveraging multi-level solar measurements, utilizing a nonlinear autoregressive with exogenous input (NARX) model. By incorporating solar measurements from diverse locations, including customer, feeder, and substation levels, their model aimed to improve the accuracy of solar photovoltaic (PV) generation forecasts. The study compared the performance of the proposed model with two-level and single-level studies, demonstrating the advantages of incorporating multi-level measurements for enhanced forecasting accuracy.

\citet{Perera_2024} focused on day-ahead regional solar power forecasting by proposing two deep-learning-based methods that effectively leverage aggregated and individual power generation time series with weather data. They introduced two hierarchical temporal convolutional neural network (HTCNN) architectures and two strategies to adapt HTCNNs for regional solar power forecasting. Their work involved evaluating the proposed approaches using a large dataset collected over a year from 101 locations across Western Australia to provide a day-ahead forecast at an hourly time resolution. The results demonstrated the effectiveness of HTCNNs in reducing regional forecast errors and requiring fewer individually trained networks compared to alternative methods.

\citet{Perera_2024} tackled the limitations of offline learning in deep learning models by presenting an Adaptive LSTM (AD-LSTM) framework for day-ahead photovoltaic power generation forecasting. The AD-LSTM model dynamically learns from new data while preserving knowledge from historical data, making it adaptable to changes in the PV system and resilient to concept drift. The study evaluated the AD-LSTM model using multiple datasets from PV systems and demonstrated its superior forecasting accuracy compared to offline LSTM and other traditional machine learning and statistical models.

\citet{Dao_2020} investigated the application of ensemble methods for enhancing short-term and medium-term solar and photovoltaic power prediction. Their research encompassed a hierarchical structure for solar radiation prediction, utilizing machine learning techniques for data clustering before applying ensemble methods. They also explored different time series models and the integration of weather forecast services to improve the prediction accuracy. The implementation of these ensemble methods on a low-cost Raspberry Pi platform demonstrated the feasibility of their approach in real-world scenarios.

\citet{Tang_2018} proposed a Least Absolute Shrinkage and Selection Operator (LASSO)-based forecasting model for solar power generation based on historical weather data. Their approach involved developing an algorithm that maximizes Kendall's tau coefficient to estimate prediction model coefficients, leveraging LASSO's variable selection capability to balance prediction accuracy and model complexity. The study evaluated the LASSO-based scheme with real-world datasets and found it to outperform existing methods, demonstrating its effectiveness in solar power generation forecasting.

Several other studies have explored day-ahead PV power forecasting using various methodologies. \citet{Gigoni_2018} conducted an extensive comparison of simple and sophisticated forecasting methodologies across 32 photovoltaic plants to evaluate the impact of weather conditions and forecasts on prediction accuracy. \citet{Conte_2020} focused on day-ahead and intra-day planning of integrated battery energy storage systems (BESS) and photovoltaic systems for frequency regulation, taking into account uncertainties in photovoltaic generation and frequency dynamics. \citet{jiang2023dayahead} proposed a day-ahead PV power forecasting method based on multiple seasonal-trend decomposition using LOESS (MSTL) and temporal fusion transformers (TFT), achieving improved accuracy compared to existing methods on a desert knowledge Australia solar centre dataset.

On an other field than power prediction, \citet{Gao_2023} published a study on trend-based stock price prediction method that employs the K-nearest neighbors (kNN) algorithm for trend forecasting. Experiments were conducted using a historical stock price dataset, and the prediction performance was evaluated. Evidences suggests that, in relation to accuracy in stock price prediction, the trend-based kNN algorithm exhibits superior performance over conventional machine learning approaches. In addition, the impact of prediction time span on model performance was investigated. The findings suggest that the trend-based kNN algorithm exhibits clear advantages when dealing with predictions over larger time spans.

The present study contributes to this body of research by focusing on univariate and multivariate day-ahead photovoltaic DC power prediction using LASSO and k-nearest neighbors (kNN) algorithms. While these algorithms have been explored in various contexts, their application to DC power prediction and comparison in a univariate and multivariate setting with one-day lag is novel. This research aims to provide insights into the comparative performance of LASSO and kNN for this specific prediction task, potentially offering valuable information for power system operators and renewable energy stakeholders.