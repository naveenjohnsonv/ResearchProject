% Conclusion
\section{Conclusion}
Based on the comparative analysis of machine learning models for predicting solar power generation from weather conditions and lagged features, several insights have emerged:

The study evaluated LASSO regression and kNN regression across different scenarios, including univariate and multivariate setups with daily and hourly data. Results indicated that LASSO regression generally outperformed kNN in both daily predictions and hourly predictions, even if kNN hit the lowest RMSE while sacrificing the trend catch. However, both models struggled to achieve desired accuracy levels, particularly in hourly predictions where the RMSE hovered around 0.5 or higher. The best performing models (Table \ref{tab:model_comparison}) were LassoCV (RMSE of 0.538 using lagged DC power and ambient temperature) and kNN with cross-validation (RMSE of 0.529 using lagged DC power and module temperature).

Challenges such as dataset volatility and limited sample size impacted model performance, particularly in capturing finer temporal trends. Despite efforts in feature engineering and model tuning, including cross-validation and hyperparameter optimization, achieving precise predictions remained elusive.

In conclusion, while LASSO regression showed promise in daily predictions, enhancing data quality and model sophistication is necessary to improve forecasting accuracy for solar power generation. These improvements would also benefit kNN, as highlighted earlier. Future research should focus on acquiring a larger dataset, engineering domain-specific features, and exploring advanced time series models to address these challenges and further refine predictive capabilities in renewable energy forecasting.